%%%%%%%%%%%%%%%%%%%%%%%%%%%%%%%%%%%%%%%%%%%%%%%%%%%%%%%%
% Relatório do Projeto Final de TP1
%%%%%%%%%%%%%%%%%%%%%%%%%%%%%%%%%%%%%%%%%%%%%%%%%%%%%%%%%

\documentclass[12pt]{article}

\usepackage{sbc-template}
\usepackage[brazil,american]{babel}
\usepackage[utf8]{inputenc}

\usepackage{graphicx}
\usepackage{url}
\usepackage{float}
\usepackage{listings}
\usepackage{color}
\usepackage{todonotes}
\usepackage{algorithmic}
\usepackage{algorithm}
\usepackage{hyperref}
     
\sloppy


\title{Trabalho Prático\\ 
Cartola FC}

%author{Nome do Aluno, Matrícula\\
\author{Élvis Júnior, 24/1038700\\
        Gustavo Alves, 24/1020779\\
        Pedro Marcinoni, 24/1002396\\
        Grupo 1
}

\address{Dep. Ciência da Computação -- Universidade de Brasília (UnB)\\
  CIC0197 - Técnicas de Programação I\\
  \email{elvismirandajr@gmailcom, gusfring.a@gmail.com, pedroextrer@gmail.com}
}

\begin{document}
\maketitle

\selectlanguage{american}
\begin{abstract}
  Write here a short summary of the report in English.
\end{abstract}
\selectlanguage{brazil}

\begin{resumo}
  Escreva aqui um pequeno resumo do Projeto.
\end{resumo}


\section{Descrição do Problema}
\label{sec:descricao}

Descrever aqui o problema a ser resolvido. O que é o projeto? Qual a sua importância? Quais os objetivos do projeto?

\section{Definição das regras de negócio}
\label{sec:regras}

Descrever aqui as regras de negócio do projeto. O que é necessário para o funcionamento do sistema? Quais as entradas e saídas? Quais os principais componentes do sistema?

\section{Diagrama de classes final}
\label{sec:classes}

Colocar aqui o diagrama de classes final do projeto. O diagrama deve ser gerado a partir do código fonte do projeto. O diagrama deve ser legível e conter todas as classes do projeto.

\section{Telas}
\label{sec:telas}

Colocar aqui as telas do projeto. As telas devem ser geradas a partir do código fonte do projeto. As telas devem ser legíveis e conter todas as funcionalidades do projeto.

\section{Conclusão}
\label{sec:conclusao}

Colocar aqui a conclusão do projeto. O que foi aprendido? O que poderia ser melhorado? Quais as dificuldades encontradas? Quais os próximos passos?

\bibliographystyle{sbc}
\bibliography{relatorio}  %Aqui é a definição do arquivo .bib a ser usado pelas referências


\end{document}
